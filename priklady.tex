Převářnou část svahu tvoří souše. Už jen z toho důvodu je důležité chápat procesy které dávají svahu konkrétní formu a faktory, které svahy dávají mantinely jednotlivým procesům. 

Se svahy se pojí i celá řada přírodních hrozeb jako jsou například sesuvy. 

V základu můžeme rozlišovat dva typy svahů.  A to podle toho, zda jsou pokryté sedimenty, půdou (angl. soil-mantled hillslopes). Tyto svahy jsou charakteristické svým hladkým průběhem. Opakem jsou svahy tvořené jen skalním podložím, neboli strukturní svahy (angl. bedrock hillslopes). Takovéto svahy jsou typické rozeklaným povrchem.

\section{Svahové procesy}

\subsection{Difuzní procesy}
Difuzní procesy jsou charakteristické tím, že se na transportu sedimentů nepodílí soustředěný tok vody, větru či ledu. Tyto procersy zahlazují nerovnosti na svahu a postupně snižují vertikální členitost reliéfu. Důležitou roli hrají vodní kapky, které dopadají na zemský povrtch. A to zejména tam, kde vegetační kryt je velice řídký či zcela chybí (pouště, badlandy). Vodní kapka, která dopadá na ukloněný povrch vymrští částečky stejný počet letí proti a po svahu. Ty, které jsou odmrštěny proti svahu ale mají kratčí trajektorii, než ty, které letí po svahu dolů. Významnějším procesem je plošný (ronový) splach. Největší účinky má na opět na svzích, které nejsou pokryté vegetací. Posledním typem difuzního procesu je ploužení. Jedná se o velmi pomalý pohyb půdy a sedimentů po svahu dolů. PLoužení může být způsobeno celou řado různých procesů. V oblastech se permafrostem se může jednat o soliflukci,. Dále to může být způsobené růstem ledových krystalů v půdě nebo vývraty. 


\subsection{Svahové pohyby}
Klasifikace svahových pohybů. Dalo by se říct klasickou kalsifikací svahových pohybů je klasifikace Varnese (1954, 1978), která prošla celou řadou úprav. 

\subsubsection{Sesuvy}
Sesuv je výsledná forma procesu sesouvání. Při sesouvání docházi k pohybu hmoty podél jedné nebo více smykových ploch. POdle tvaru smykové plochy můžeme sesuvy rozlišit na dva základní typy. Rotační sesuv má smykovou plochu zakřivenou, válcovou. Na tvar smykové plochy nemají vliv geologické struktury (pukliny, vrstevní plochy). Rotační sesuvy jsou typické pro nezpevněné čí málo zpevněné materiály.Povrch sesunutých bloků se tak uklání proti svahu. Rotační sesuvy jsou typické pro homogenní materiály, slabé horniny a hlavně kohezivní zeminy. 

Translační nebo také planární sesuvy mají smykovou plochu rovnou. Ta je většinou predisponovaná nějakou nespojitostí v hornině. Může se jednat například o vrstevní plochy, pukliny, plochy foliace apod. Translační sesuvy ale mohou vznikat i na rozhraní sediment -- skalní podloží. Specifickým typem translačních sesuvů jsou tzv. wedge slides klínové sesuvy. Smyková plocha je tvořená zpravidla dvěmi strukturami (např. puklinami), které vytvářejí nestabilní blok.

\subsubsection{Řícení}

Řícení je nejrychlejším gravitačním procesem, kdy pohyb hmoty je alespoň v části trajektorie realizován volným pádem. Podle charakteru řícení jej můžeme klasifikovat jako úlomkové, blokové apod.

\subsubsection{Tečení}
Blokovobahenní proud, mura (debris flow)
Velice rychlý tok horninového materiálu a vody. 

\subsubsection{Odsedání}

Odsedání (angl. toppling) je proces, kdy se se horninový blok překlápí dopředu po svahu.



\subsubsection{Boční rozvolňování}

