\section{Geomorfometrie jako nástroj poznání reliéfu}
V předchozích kapitolách jsme se věnovali různým tvarům zemského povrchu (morfografii) a procesům, které je utvářejí (morfogenezi). V této kapitole bych chtěl nastínit základy (geo)morfometrie. 

Geomorfometrii pěkně definoval prof. John Lindsay: \enquote{Geomorfometrie je o porozumění krajiny prostřednictvím digitální topografie.}

Geomorfometrie se zabývá číselným vyjádřením vlastností zemského reliéfu. Můžeme ji rozdělit do dvou oblastí podle toho, jak pojímají zemský reliéf. \enquote{\emph{Konkrétní}} geomorfometrie (\textit{specific geomorphometry}) se zabývá analýzou jednotlivých forem reliéfu (moréna, sesuv, meandr...). Měříme jejich plocho, objem, půdorysný tvar, u meandrů sinusiotu apod. Druhá oblast geomorfometrie chápe zemský reliéf jako kontinuum. Nevnímáme povrch jako soubor jednotlivých tvarů, ale jako pole nadmořských výšek. Tuto oblast nazýváme \emph{obecnou} geomorfometrií (\textit{general geomorphometry}).

Geomorfometrii můžeme považovat za samostatnou disciplínu, která čerpá z celého spektra disciplín. Samozřejmě jedním pilířem jsou vědy o Zemi (geomorfologie, hydrologie...), dále pak čerpá z matematiky (geometrie, topologie, geostatistika...) a v neposlední řadě z informatiky (geografické informační systémy, zpracování obrazu, rozpoznávání vzorů...). Poznatky z těchto disciplín jsou pak využívány pro zjišťování parametrů zemského povrchu, jeho modelování a analýze. Výsledky geomorfometrických analýz pak nacházejí uplatnění v celé řadě odvětví. Jedná se samozřejmě opět o geomorfologii, hydrologii, oceánografii, své uplatnění ale nachází i v civilním inženýrství, ale i vojenství. 

V této kapitole se budu zabývat hlavně obecnou geomorfologií.

\section{Digitální výškové modely}
Výškopis je možné vyjádřit digitálně pomocí celé řady datových modelů. Může se jednat o \emph{vektorová data} (vrstevnice -- linie, výškové kóty -- body), nepravidelnou trojúhelníkovou síť \emph{TIN}, \emph{mračno bodů} (\textit{point cloud}) a v neposlední řadě rastr.

Jak v české, tak v anglické terminologii panuje poměrně chaos a do digitální výškových modelů (\textit{DEM - digital elevation model}) jsou zahrnovány jak rastrové modely, TIN ale i mračna bodů. Dle definice \textcite{guthDigitalElevationModels2021} je digitální výškový model (DEM) digitální reprezentací nadmořských výšek v pravoúhlém gridu. Alternativní vyjádření výšek pomocí TINu, mračna bodů či vrstevnic není dle této definice DEM, neboť se nejedná o pravoúhlou síť (matici) nadmořských výšek.

	 
 %Rozvoj výpočetní techniky zasáhl významně i geomorfologii. V současné době se geomorfologický výzkum bez digitálních modelů terénu takřka neobejde. 

%Pojetí digitálních výškových modelů DEM

\section{Geomorfometrické parametry}
Základním parametrem je (nadmořská) výška. Veškeré ostatní parametry jsou nějakým způsobem od výšky odvozené.  

\emph{Lokální morfometrické parametry} jsou počítané z malého okolí a jsou vztažené k centrální buňce. Pro jejich výpočet není zapotřebí celého území (rastru). Příkladem takové morfometrické parametru je například sklon.
Pro výpočet tzv. \emph{regionálních morfometrických parametrů} je zapotřebí všech pixelů digitálního modelu. Zpravidla se jedná o parametry, které souvisí zejména s tokem hmoty (vody) v krajině (akumulace toku apod.)  

\section{Nástroje pro geomorfometrické analýzy}
Základní morfometrické analýzy jako je výpočet sklonu, orientace, některé druhy zakřivení reliéfu (vrstevnicová a spádnicová křivost) lze provádět ve všech klasických geoinformačních systémech (GIS), které jsou jak komerční (Arc GIS Pro, Arc Map), tak i volně dostupná jako open source QGIS, SAGA GIS, GRASS GIS.  

Mimo tyto nástroje se stále více uplatňují i externí knihovny geomorfometrických nástrojů, které rozšiřují možnosti výše uvedených programů, či jsou přímo dělány pro využití při práci v programovém jazyce Python či R.

 