\section{Geomorfometrie jako nástroj poznání reliéfu}
V předchozích kapitolách jsme se věnovali různým tvarům zemského povrchu (morfografii) a procesům, které je utvářejí (morfogenezi). V této kapitole bych chtěl nastínit základy (geo)morfometrie. 

Geomorfometrie se zabývá číselným vyjádřením vlastností zemského reliéfu. Můžeme ji rozdělit do dvou oblastí podle toho, jak pojímají zemský reliéf. \enquote{\emph{Konkrétní}} geomorfometrie (\textit{specific geomorphometry}) se zabývá analýzou jednotlivých forem reliéfu (moréna, sesuv, meandr...). Měříme jejich plocho, objem, půdorysný tvar, u meandrů sinusiotu apod. Druhá oblast geomorfometrie chápe zemský reliéf jako kontinuum. Nevnímáme povrch jako soubor jednotlivých tvarů, ale jako pole nadmořských výšek. Tuto oblast nazýváme \emph{obecnou} geomorfometrií (\textit{general geomorphometry}).

Geomorfometrii můžeme považovat za samostatnou disciplínu, která čerpá z celého spektra disciplín. Samozřejmě jedním pilířem jsou vědy o Zemi (geomorfologie, hydrologie...), dále pak čerpá z matematiky (geometrie, topologie, geostatistika...) a v neposlední řadě z informatiky (geografické informační systémy, zpracování obrazu, rozpoznávání vzorů...). Poznatky z těchto disciplín jsou pak využívány pro zjišťování parametrů zemského povrchu, jeho modelování a analýze. Výsledky geomorfometrických analýz pak nacházejí uplatnění v celé řadě odvětví. Jedná se samozřejmě opět o geomorfologii, hydrologii, oceánografii, své uplatnění ale nachází i v civilním inženýrství, ale i vojenství. 

V této kapitole se budu zabývat hlavně obecnou geomorfologií.

\section{Digitální výškové modely}
Digitální výškové modely (\textit{digital elevation models, DEM}) jsou základem pro veškeré geomorfometrické analýzy. Výškopis je možné vyjádřit digitálně pomocí celé řady datových modelů. Může se jednat o vektorová data (vrstevnice -- linie, výškové kóty -- body), nepravidelnu trojúhelníkovou síť TIN, mračno bodů (\textit{point cloud}) a v neposlední řadě rastr. %Rozvoj výpočetní techniky zasáhl významně i geomorfologii. V současné době se geomorfologický výzkum bez digitálních modelů terénu takřka neobejde. 

%Pojetí digitálních výškových modelů DEM





\section{Geomorfometrické parametry}





\subsection{Lokální morfometrické parametry}




\subsection{Regionální morfometrické parametry}




\section{Nástroje pro geomorfometrické analýzy}