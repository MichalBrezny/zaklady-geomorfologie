\begin{table*}[]
	\small
	\begin{tabularx}{1\textwidth}{@{}lp{2cm}p{2cm}X@{}}
		\toprule
		Morfoklimatická zóna       & Průměrná roční teplota & Průměrné roční srážky & Důležitost geomorfologických procesů                                           \\ \midrule
		Humidní tropická           & $20-30$                 & $>1500$                 & Velká úroveň chemického zvětrávání, mechanické zvětrávání je omezené,          \\
		Tropické (období deště, sucha) &
		$20-30$ &
		$600-1500$ &
		Chemické zvětrávání aktivní během vlhkých period, mechanické zvětrávání nízké až střední \\
		Tropická semi-aridní       & $10-30$                  & $300-600$               & Chemické zvětrávání střední až nízké, mechanické zvětrávání je lokálně aktivní \\
		Tropické aridní            & $10-30$                  & $0-300$                 & Mechanické zvětrávání vysoké, minimální chemické zvětrávání                    \\
		Humidní střední zem. šířky & $0-20$                   & $400-1800$              & Chemické zvětrávání střední,                                                   \\
		Suché kontinentální        & $0-10$                   & $100-400$               & Chemické zvětrávání nízké až střední                                           \\
		Periglaciální              & $<0$                     & $100-1000$              & Velice silné mechanické zvětrávání, chemické zvětrávání je nízké až střední    \\
		Glaciální                  & $<0$                     & $0-1000$                & Vysoké mechanické zvětrávání, chemické zvětrávání nízké                        \\
		Azonální horské zóny &
		Vysoce variabilní &
		Vysoce variabilní &
		Rychlosti všech procesů se značně mění s nadmořskou výškou; mechanická, glaciální eroze je výrazná ve vysokých nadmořských výškách \\ \bottomrule
	\end{tabularx}
	\caption{Hlavní morfoklimatické zóny (Upraveno podle \textcite{summerfieldGlobalGeomorphologyIntroduction1999})}
	\label{tab:morfoklimaticke}
\end{table*}
