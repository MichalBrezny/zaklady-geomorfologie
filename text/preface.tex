\chapter*{Předmluva}
% Zemský povrch -- hranice mezi tím, co je pod zemí a atmosférou. 
Pro nezasvěceného člověka je možná až nepředstavitelné, jak moc zemský reliéf ovlivňoval a ovlivňuje lidskou společnost. Vzpomeňte si na starověký Egypt a jeho závislost na pravidelných Nilských záplavách. Když chceme navštívit zříceninu hradu, tak zpravidla musím stoupat na špatně dostupný kopec, který v dobách minulých umožňoval snadnou obranu a poskytoval rozhled do širokého okolí. Vinice jsou u nás na svazích s jižní orientací -- tak aby byly co nejlépe osluněné. 

Ovlivňování zemského reliéfu je ale obousměrné. Člověk svým působením v krajině reliéf ve velké míře přetváří. Těžíme a přesouváme obrovské objemy hmot a vytváříme tak nové tvary -- lomy, náspy, zemědělské terasy. Stačí se zajet podívat na hnědouhelné lomy na Mostecku nebo haldy na Ostravsku.  Reliéf ale přetváříme i nepřímo. Ovlivňujeme tok látek v krajině. Narovnáním vodních toků jsme urychlili odtok vody v krajině. Stavbou přehrad kromě možnosti regulace množství vody v řece znemožňujeme pohyb sedimentů. A řeky na to reagují zahlubováním. Urychlujeme přírodní procesy (odlesněním krajiny zrychlujeme půdní erozi) nebo se naopak snažíme erozi zpomalit např. vytvářením různých zasakovacích pásů. Omezujeme řeky v meandrování opevňováním břehů. V současné době se lidská společnost podílí na přesunech hmoty v podobných dimenzích jako přírodní procesy. 

Zemský reliéf ovlivňuje i celou řadu dalších složek krajiny. Na svazích budete mít menší půdní pokryv než na úpatí. S rostoucí nadmořskou výškou se snižuje teplota a roste množství srážek, což se odrazí i na vegetaci. Pohoří tvoří bariéry pro vzduchové masy a vlhkost. Podívejte se například na Andy -- západní strana je jednou ze srážkově nejbohatších oblastí na Zemi, Patagonie naopak leží ve srážkovém stínu And. Roční úhrny srážek dosahují maximálně \qtyrange{200}{300}{\milli\metre}. 

Celá řada přírodních procesů, které utvářejí zemský reliéf mohou být i přírodní hrozbou. Povodně, zemětřesení, sesuvy, písečné bouře každoročně mají na svědomí velké množství lidských životů a obrovské ekonomické škody. 

Studiem tvarů reliéfu a procesů, které ho utvářejí se zabývá geomorfologie. V současně době již objektem studia není jen reliéf planety Země, ale pozornost se obrací i k dalším planetám Sluneční soustavy. Zkoumáním forem reliéfu se totiž můžeme dozvědět velké množství informací o procesech současných a minulých.  

Jak jsem se snažil v předchozích odstavcích naznačil, reliéf a reliéfotvorné procesy se projevují v mnoha oblastech. Myslím si, že to je docela pádný důvod pro to, aby každý měl alespoň základní ponětí o procesech, tvarech reliéfu a vazbách na další složky krajiny. Věřím, že Vám tato učebnice s pomůže s poznáním geomorfologie. 

\newpage
Tato učebnice si neklade za cíl obsáhnout všechna zákoutí geomorfologie a přinést vyčerpávající přehled tvarů reliéfu. Nebylo by to ani v mých silách, neboť geomorfologie je velice široký obor. Mým cílem bylo vytvořit učebnici, která uceleně shrne základy geomorfologie a geomorfologických procesů pro studenty fyzické geografie, učitelství geografie ale i další příbuzných oborů. Doufám ale, že bude zajímavým čtením pro všechny, co se chtějí něco dozvědět krajině. 

Stejně jako reliéf není v čase statický a nějakým způsobem se vyvíjí, tak i tato učebnice se bude vyvíjet. Budu rád, když mi napíšete jakékoliv postřehy k obsahu, upozorníte mě na chyby, nejasná či zavádějící vyjádření, či mi dáte tipy k doplnění.

Učebnici tvořím podobně jako se tvoří programy, postupně vytvářím nové verze učebnice podle toho jak přidávám a upravuji obsah. Na mém GitHubu (\url{https://github.com/MichalBrezny/zaklady-geomorfologie}) najdete pod \enquote{release} PDF aktuální verzi učebnice, která odpovídá tomu, co je v \enquote{main} větvi repositáře. V \enquote{pracovni} větvi je text učebnice, který jsem ještě neuzavřel a pracuje se na něm. Můžete do něj ale nahlédnout. Na GitHubu mi můžete psát i připomínky a najdete tam můj TODO list. 

\vspace{2cm}
Přeji příjemné a věřím, že i poučné čtení.
\begin{flushright}
	Michal Břežný
\end{flushright}

 
%
%
%%