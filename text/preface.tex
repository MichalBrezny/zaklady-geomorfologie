\chapter*{Předmluva}
Zemský povrch. Hranice mezi tím, co je pod zemí a atmosférou. Pro nezasvěceného člověka je možná až nepředstavitelné, jak zemský reliéf, jeho rozličné tvary a parametry ovlivňují lidskou společnost, chování či hospoodaření. Toto ovlivňování je ale obousměrné. Člověk svým působením v krajině reliéf ve velké míře přetváří. vytváříme nové tvary -- například lomy, náspy, zemědělské terasy. Reliéf ale přetváříme i nepřímo. Ovlivňujeme tok látek v krajině (například narovnáváním vodních toků, stavbou přehrad), urychlujeme přírodní procesy (odlesněním krajiny zrychlujeme půdní erozi) nebo se naopak snažíme erozi zpomalit. 

Studiem tvarů reliéfu a procesů, které ho utvářejí se zabývá geomorfologie. V současně době již objektem studia není jen reliéf planety Země, ale pozornost se obrací i k dalším planetám Sluneční soustavy. Zkoumáním forem reliéfu se totiž můžeme dozvědět velké množství informací o procesech současných a minulých.  

Učebnice, která se vám dostala do ruky si dává za cíl představit srozumitelnou cestou fascinující svět geomorfologie. 

Učebnici jsem začal psát jako skripta pro studenty učitelství geografie a rozhodl se rozšířit to pro všechny studenty geografie, fyzické geografie a příbuzných oborů. 

Stejně jako reliéf není v čase statický a nějak se vyvíjí, i tato učebnice se bude vyvíjet. Budu rád, když mi napíšete jakékoliv postřehy k obsahu, upozorníte mě na chyby, nejasná či zavádějící vyjádření.

 
%
%
%%